\documentclass[a4paper, 12pt]{exam}
\usepackage[T1]{fontenc}
\usepackage{amsmath}
\usepackage{amssymb}
\usepackage{enumerate}
\usepackage{bm}
\usepackage{advdate}
\usepackage{datetime}
\usepackage[mathcal]{eucal}
\usepackage{dsfont}
\usepackage[numbered,framed]{matlab-prettifier}
\usepackage{url}
\newdate{issuedate}{25}{9}{2020}
\newdate{duedate}{9}{10}{2020}

% \newcommand{\duedate}[1][14]{%
% \begingroup
% \AdvanceDate[#1]%
% \today%
% \endgroup
% }%

\usepackage[thehwcnt=1]{iidef}
\thecourseinstitute{Tsinghua-Berkeley Shenzhen Institute}
\thecoursename{Learning From Data}
\theterm{Fall 2020}
\makeatletter
\newcommand{\firstblock}{programming_policies}
\makeatother

\begin{document}

\pagestyle{headandfoot}
\runningheadrule


\newcounter{psctr}
\setcounter{psctr}{1} % set to the times of problem

\runningheader{Programming Assignment \thepsctr}
              {\textsc{Learning from Data}}
              { Page \thepage\ of \numpages}
\firstpagefooter{}{}{}
\runningfooter{}{}{}


\newcounter{Sequ}
\newenvironment{Sequation}
   {\stepcounter{Sequ}%
     \addtocounter{equation}{-1}%
     \renewcommand\theequation{S\arabic{Sequ}}\equation}
   {\endequation}
%\topskip0pt

% \vspace*{\fill}
\centering

% \vspace{0.3em}
\centering
\renewcommand{\thequestion}{\arabic{psctr}.\arabic{question}}
\hwname{Programming Assignment}
\courseheader
\begin{flushleft}
  \textbf{Issued:} \displaydate{issuedate} \hfill
  \textbf{Due:} \displaydate{duedate} 
\end{flushleft}

\hrule 

\input{\firstblock}

%\pointname{}
%\vspace{\footskip}
\vspace{1em}


%\pointname{}
%\vspace{\footskip}
%\vspace{1em}

\begin{questions}
\question (5 points) \emph{Linear regression with regulation} You have learned in class the linear observation model
\begin{equation*}
\bm{y} = X\bm{w} + \bm{c}
\end{equation*}
where the $X$ is the observation matrix;
$\bm{w}$ is the weight vector to be estimated and $\bm{c}$ is the Gaussian noise.

We know that ordinary linear regression uses the formula $\bm{w} = (X^TX)^{-1}X^T\bm{y}$ to calculate the weight, which is the minimizer
of $||\bm{y} - X\bm{w}||^2$.
When the matrix $X^TX$ is near singular, ordinary regression does not work well. This happens when
the data suffer from multicollinearity. Multicollinearity means there exists near-linear relationships among the independent variables.
For example, when the columns of matrix $X$ are dependent, then $X^TX$ is singular. To overcome this problem, we try to solve an alternative optimization problem
$||\bm{y} - X\bm{w}||^2 + \alpha ||\bm{w}||^2$ where $\alpha \geq 0$ is a hyper-parameter. We call the method with extra $\alpha$ as Ridge Regression.

Please implement Ridge regression. See the details in the \textbf{ridge\_regression.py}.
Your implementation should handle the multicollinearity case.

Hint: Minimize the loss $||\bm{y} - X\bm{w}||^2 + \alpha ||\bm{w}||^2$ is equivalent with solving the equation
$( X^T X+ \alpha I)\bm{w}  = X^T \bm{y}$.

\question (5 points) \emph{Logistic regression with Newton's method} You have learned in class that using maximal likelihood to estimate the parameters of the logistic regression model is equivalent to maximize:
\begin{equation*}
\mathcal{L}(w) = \sum_{i=1}^n y_i \bm{w}^T \bm{x}_i - \log ( 1 + \exp(\bm{w}^T \bm{x}_i))
\end{equation*}
We can use Newton's method to find the optimal $\bm{w}$, which uses the following update scheme for $\bm{w}$:
\begin{equation*}
w_{t+1} \leftarrow w_t - H^{-1} \nabla_w \mathcal{L}(w)|_{w_t}
\end{equation*}
where $H$ is the Hessian matrix for the likelihood function $\mathcal{L}$.
The scheme can be written in compact form as:
\begin{equation*}
\bm{w}_{t+1} = \bm{w}_t + (X^TRX)^{-1} X^T(\bm{y}-\bm{u}), \textrm{ where } \mu_i = \frac{1}{1+\exp(-\bm{w}_i^T\bm{x}_i)} \textrm{ and } R_{ii} = \mu_i ( 1 - \mu_i)
\end{equation*}
where $R$ is a diagonal matrix. Using Newton's method to fit the logistic model is also called Iterative Reweighted Least Sqaures (IRLS).

Please implement IRLS. See details in \textbf{logistic\_regression.py}.

\end{questions}


\nocite{*}
\begin{flushleft}
\textbf{Notice}: \\
\begin{enumerate}
\item Use matrix operations other than loops for efficiency. If the running time of Auto-Grading steps exceeds 5 minutes, you will get point deductions.
\item You are expected to only use \texttt{numpy} packages to implement the algorithms.
\item All questions assume that the data are centered around zero. Therefore, you do not need to train the extra bias parameter in your code.
\end{enumerate}
\end{flushleft}

%\bibliographystyle{plain}
%\bibliography{ref}
%\begin{thebibliography}{9}
%	\bibitem{ridge} \href{https://ncss-wpengine.netdna-ssl.com/wp-content/themes/ncss/pdf/Procedures/NCSS/Ridge_Regression.pdf}{Ridge Regression}
%	\bibitem{tutorial} \href{https://www.datacamp.com/community/tutorials/tutorial-ridge-lasso-elastic-net}{Regularization: Ridge, Lasso and Elastic Net}
%\end{thebibliography}
\end{document}