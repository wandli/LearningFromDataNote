\documentclass[a4paper, 12pt]{exam}
\usepackage[T1]{fontenc} 
\usepackage{amsmath}
\usepackage{amssymb}
\usepackage{enumerate}
\usepackage{bm}
\usepackage{advdate}
\usepackage{datetime}
\usepackage[mathcal]{eucal}
\usepackage{dsfont}
\usepackage[numbered,framed]{matlab-prettifier}
\usepackage{url}
\newdate{issuedate}{25}{9}{2020}
\newdate{duedate}{9}{10}{2020}

% \newcommand{\duedate}[1][14]{%
% \begingroup
% \AdvanceDate[#1]%
% \today%
% \endgroup
% }%

\usepackage[thehwcnt=1]{iidef}
\thecourseinstitute{Tsinghua-Berkeley Shenzhen Institute}
\thecoursename{Learning From Data}
\theterm{Fall 2020}
\makeatletter
\newcommand{\firstblock}{programming_policies}
\makeatother

\begin{document}
	
	\pagestyle{headandfoot}
	\runningheadrule
	
	
	\newcounter{psctr}
	\setcounter{psctr}{1} % set to the times of problem
	
	\runningheader{Programming Assignment \thepsctr}
	{\textsc{Learning from Data}}
	{ Page \thepage\ of \numpages}
	\firstpagefooter{}{}{}
	\runningfooter{}{}{}
	
	
	\newcounter{Sequ}
	\newenvironment{Sequation}
	{\stepcounter{Sequ}%
		\addtocounter{equation}{-1}%
		\renewcommand\theequation{S\arabic{Sequ}}\equation}
	{\endequation}
	%\topskip0pt
	
	% \vspace*{\fill}
	\centering
	
	% \vspace{0.3em}
	\centering
	\renewcommand{\thequestion}{\arabic{psctr}.\arabic{question}}
	\hwname{Programming Assignment}
	\courseheader
	\begin{flushleft}
		\textbf{Issued:} \displaydate{issuedate} \hfill
		\textbf{Due:} \displaydate{duedate} 
	\end{flushleft}
	
	\hrule 
	
	\input{\firstblock}
	
	%\pointname{}
	%\vspace{\footskip}
	\vspace{1em}
	
	
	%\pointname{}
	%\vspace{\footskip}
	%\vspace{1em}
	
	\begin{questions}
		\question (5 points) \emph{Linear regression with regularization.} In class, we have learned linear regression using the linear model
		\begin{equation*}
		\bm{y} = \bm{X}\bm{\theta} + \bm{\epsilon}
		\end{equation*}
		where $\bm{X}$ is the observation matrix,
		$\bm{\theta}$ is the weight vector to be estimated and $\bm{\epsilon}$ is the Gaussian noise. To minimize the noise effect in the model, we do a minimization problem
		\begin{equation*}
		\min_{\theta}||\bm{y} - \bm{X}\bm{\theta}||^2
		\end{equation*}
		and get the best weight estimator
		$\argmin_{\theta}||\bm{y} - \bm{X}\bm{\theta}||^2 = (\bm{X}^T\bm{X})^{-1}\bm{X}^T\bm{y}$. This process may be problematic when matrix $\bm{X}^T\bm{X}$ is nearly singular, which means it has very small singular values. %This happens when the data suffer from multicollinearity. Multicollinearity means there exists near-linear relationships among the feture vectors (i.e. column vectors of $X$).
		%For example, when the columns of matrix $X$ are linear dependent, then $X^TX$ is singular. 
		To deal with this problem, we add a regularization in linear model and we have
		\begin{equation*}
		\min_{\theta} ||\bm{y} - X\bm{\theta}||^2 + \alpha ||\bm{\theta}||^2
		\end{equation*}
		where $\alpha \geq 0$ is a hyper-parameter. The method with extra $\alpha$ is also known as {em ridge regression}.
		
		Please implement ridge regression  by completing the code in  \textbf{ridge\_regression.py}.
		Your implementation should handle the case when $\alpha=0$.
		
		Hint: Minimizing the loss $||\bm{y} - X\bm{\theta}||^2 + \alpha ||\bm{\theta}||^2$ is equivalent to solving the equation
		$( X^T X+ \alpha I)\bm{\theta}  = X^T \bm{y}$.
		
		\question (5 points) \emph{Logistic regression with Newton's method. } You have learned in class that using maximal likelihood to estimate the parameters of the logistic regression model is equivalent to maximizing:
		\begin{equation*}
		\mathcal{L}(w) = \sum_{i=1}^n y_i \bm{\theta}^T \bm{x}^{(i)} - \log ( 1 + \exp(\bm{\theta}^T \bm{x}^{(i)}))
		\end{equation*}
		We can use Newton's method to find the optimal $\bm{\theta}$, which has the following update scheme for $\bm{\theta}$:
		\begin{equation*}
		\bm{\theta}_{t+1} \leftarrow \bm{\theta}_t - H^{-1} \nabla_{\bm{\theta}} \mathcal{L}(\bm{\theta})|_{\bm{\theta}_t}
		\end{equation*}
		where $H$ is the Hessian matrix for the likelihood function $\mathcal{L}$.
		The scheme can be written in compact form as:
		\begin{equation*}
		\bm{\theta}_{t+1} = \bm{\theta}_t + (X^TRX)^{-1} X^T(\bm{y}-\bm{\mu}), \textrm{ where } \mu_i = \frac{1}{1+\exp(-\bm{\theta}_t^T\bm{x}^{(i)})} \textrm{ and } R_{ii} = \mu_i ( 1 - \mu_i)
		\end{equation*}
		where $R$ is a diagonal matrix. Using Newton's method to fit the logistic model is also called {\em iterative reweighted least sqaures} (IRLS).
		
		Please implement IRLS by completing the code in \textbf{logistic\_regression.py}.
		
	\end{questions}
	
	
	\nocite{*}
	\begin{flushleft}
		\textbf{Notice}: \\
		\begin{enumerate}
			\item Use matrix operations other than loops for efficiency. If the running time of Auto-Grading steps exceeds 5 minutes, you will get point deductions.
			\item You are expected to only use \texttt{numpy} packages to implement the algorithms.
			\item All questions assume that the data are centered around zero. Therefore, you do not need to train the extra bias parameter in your code.
		\end{enumerate}
	\end{flushleft}
	
	%\bibliographystyle{plain}
	%\bibliography{ref}
	%\begin{thebibliography}{9}
	%	\bibitem{ridge} \href{https://ncss-wpengine.netdna-ssl.com/wp-content/themes/ncss/pdf/Procedures/NCSS/Ridge_Regression.pdf}{Ridge Regression}
	%	\bibitem{tutorial} \href{https://www.datacamp.com/community/tutorials/tutorial-ridge-lasso-elastic-net}{Regularization: Ridge, Lasso and Elastic Net}
	%\end{thebibliography}
\end{document}