\documentclass{article}
\usepackage{amsmath}
\begin{document}
(cartpole) Consider an inverted pendulum tied to a cart, our goal is to stabilize it by moving the cart
to the left or to the right.

Now we formulate the problem in classicial mechanics and Reinforcement Learning terminology.
We choose the length of the rod $L = 0.5 m$, the mass of the cart $M=1 kg$, the mass of the ball $m=0.1 kg$
and there is no friction between the cart and the floor.
The state of the system is described by
$(x, \theta, \dot{x}, \dot{\theta})$. Notice that $(x, \dot{x})$ is the position of velocity of the cart while
$(\theta, \dot{\theta})$ is the relative position and angular velocity of the ball with respect to the cart.

The action we can do is applying a force
$F$ or $-F$. We specify that the right direction is positive. To make the problem concrete,
we require $F=10N$ and we discrete the time such that at time $t+\Delta t$ we can choose another action.
$\Delta t = 0.02s$. The new state $(x', \theta', \dot{x}', \dot{\theta}')$ is updated from old state by the following law:
\begin{align*}
x' &= x + \Delta t \cdot \dot{x} \\
\theta' &= \theta + \Delta t \cdot \dot{\theta} \\
\dot{x}' &= \dot{x} + \Delta t \cdot \ddot{x} \\
\dot{\theta}' &= \dot{\theta} + \Delta t \cdot \ddot{\theta}
\end{align*}
The acceleration $(\ddot{x}, \ddot{\theta})$ can be computed from Newton's law \cite{cartpole}.
We list its equation here for convenience:
\begin{align*}
& (m + M)\ddot{x} + mL(\ddot{\theta} \cos \theta - \dot{\theta}^2 \sin \theta) x= F \\
& m l \ddot{\theta} + m \ddot{x} \cos\theta = m g \sin \theta \\
\end{align*}

\begin{thebibliography}{9}
	\bibitem{cartpole} Florian, Razvan V. "Correct equations for the dynamics of the cart-pole system." Center for Cognitive and Neural Studies (Coneural), Romania (2007).
\end{thebibliography}
\end{document}