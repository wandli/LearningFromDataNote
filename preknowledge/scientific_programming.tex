\documentclass[handout]{beamer} %

 
\mode<presentation>
{
%\usetheme[titleprogressbar]{m}
  \usetheme{default}      % or try Darmstadt, Madrid, Warsaw, ...
  \usecolortheme{default} % or try albatross, beaver, crane, ...
  \usefonttheme{default}  % or try serif, structurebold, ...
  \setbeamertemplate{navigation symbols}{}
  \setbeamertemplate{caption}[numbered]
} 

\setbeamerfont{section in toc}{size=\Large}
\setbeamerfont{subsection in toc}{size=\large}
\addtobeamertemplate{navigation symbols}{}{%
    \usebeamerfont{footline}%
    \usebeamercolor[fg]{footline}%
    \hspace{1em}%
    \insertframenumber/\inserttotalframenumber
}

\usepackage{scrextend}

\usepackage[english]{babel}
\usepackage{color}
\usepackage[utf8x]{inputenc}
\usepackage{booktabs}
\usepackage{colortbl}
\usepackage{verbatim}
\usepackage{bm}
\usepackage{listings}
\usepackage{amsmath} 
\usepackage{hyperref}
\usepackage{array}
\usepackage{wasysym}
\usepackage{xmpmulti}
\usepackage{url}
%\newcounter{propCounter} 

\newtheorem{prop}{Proposition}
 \setbeamertemplate{theorems}[numbered]


\makeatletter
\renewenvironment{proof}[1][\proofname]{\par
  \pushQED{\qed}%
  \normalfont \topsep6\p@\@plus6\p@\relax
  \trivlist
  \item[\hskip\labelsep
        \itshape
    #1\@addpunct{.}]\mbox{} 
}{%
  \popQED\endtrivlist\@endpefalse
}
 
\newenvironment<>{proofs}[1][\proofname]{%
    \par
    \def\insertproofname{#1\@addpunct{.}}%
    \usebeamertemplate{proof begin}#2}
  {\usebeamertemplate{proof end}}
 \newenvironment<>{proofe}{%
    \par
    \pushQED{\qed}
    \setbeamertemplate{proof begin}{\begin{block}{}}
    \usebeamertemplate{proof begin}}
  {\popQED\usebeamertemplate{proof end}}

\makeatother




\def\blockSqz{\vspace*{-\baselineskip}\setlength\belowdisplayshortskip{0pt}}
\def\Put(#1,#2)#3{\leavevmode\makebox(0,0){\put(#1,#2){#3}}}
\usepackage{mathtools}
\hypersetup{colorlinks=false,linkcolor=green,pdfborderstyle={/S/U/W 1}}
\usepackage{tikz}
\def\checkmark{\tikz\fill[scale=0.4](0,.35) -- (.25,0) -- (1,.7) -- (.25,.15) -- cycle;} 
\usetikzlibrary{decorations.pathreplacing,calc}

\newcommand{\tikzmark}[2][-3pt]{\tikz[remember picture, overlay, baseline=-0.5ex]\node[#1](#2){};}

\tikzset{brace/.style={decorate, decoration={brace}},
 brace mirrored/.style={decorate, decoration={brace,mirror}},
}

\newcounter{brace}
\setcounter{brace}{0}
\newcommand{\drawbrace}[3][brace]{%
 \refstepcounter{brace}
 \tikz[remember picture, overlay]\draw[#1] (#2.center)--(#3.center)node[pos=0.5, name=brace-\thebrace]{};
}

\newcounter{arrow}
\setcounter{arrow}{0}
\newcommand{\drawcurvedarrow}[3][]{%
 \refstepcounter{arrow}
 \tikz[remember picture, overlay]\draw (#2.center)edge[#1]node[coordinate,pos=0.5, name=arrow-\thearrow]{}(#3.center);
}

% #1 options, #2 position, #3 text 
\newcommand{\annote}[3][]{%
 \tikz[remember picture, overlay]\node[#1] at (#2) {#3};
}
\usepackage{ellipsis} \renewcommand{\ellipsisgap}{0.05em}
\lstset{ 
  basicstyle=\ttfamily\footnotesize
}
\def\a{\alpha}
\def\dash{\text{ --- }}
\def\E{\mathbb{E}}
\def\one{\mathbf{1}}
\def\xit[#1]{{(x^{({#1})})}^T}%{(x^{(1)})}^T }
\DeclareMathOperator*{\argmin}{arg\!\min}
\DeclareMathOperator*{\argmax}{\arg\!\max} 
\usepackage{color} 
\usepackage{listings} 
\usepackage{setspace} 

\definecolor{Code}{rgb}{0,0,0} 
\definecolor{Decorators}{rgb}{0.5,0.5,0.5} 
\definecolor{Numbers}{rgb}{0.5,0,0} 
\definecolor{MatchingBrackets}{rgb}{0.25,0.5,0.5} 
\definecolor{Keywords}{rgb}{0,0,1} 
\definecolor{self}{rgb}{0,0,0} 
\definecolor{Strings}{rgb}{0,0.63,0} 
\definecolor{Comments}{rgb}{0,0.63,1} 
\definecolor{Backquotes}{rgb}{0,0,0} 
\definecolor{Classname}{rgb}{0,0,0} 
\definecolor{FunctionName}{rgb}{0,0,0} 
\definecolor{Operators}{rgb}{0,0,0} 
\definecolor{Background}{rgb}{0.98,0.98,0.98} 
 
\lstdefinelanguage{Python}{ 
numbers=left, 
numberstyle=\footnotesize, 
numbersep=1em, 
xleftmargin=1em, 
framextopmargin=2em, 
framexbottommargin=2em, 
showspaces=false, 
showtabs=false, 
showstringspaces=false, 
frame=l, 
tabsize=4, 
% Basic 
basicstyle=\ttfamily\small\setstretch{1}, 
backgroundcolor=\color{Background}, 
% Comments 
commentstyle=\color{Comments}\slshape, 
% Strings 
stringstyle=\color{Strings}, 
morecomment=[s][\color{Strings}]{"""}{"""}, 
morecomment=[s][\color{Strings}]{'''}{'''}, 
% keywords 
morekeywords={import,from,class,def,for,while,if,is,in,elif,else,not,and,or,print,break,continue,return,True,False,None,access,as,,del,except,exec,finally,global,import,lambda,pass,print,raise,try,assert}, 
keywordstyle={\color{Keywords}\bfseries}, 
% additional keywords 
morekeywords={[2]@invariant,pylab,numpy,np,scipy}, 
keywordstyle={[2]\color{Decorators}\slshape}, 
emph={self}, 
emphstyle={\color{self}\slshape}, 
% 
} 
\title[Learning From Data]{Learning From Data \\ Review Session:  Scientific Programming in Python  }

%\author{Shao-Lun Huang  \quad shaolun.huang@sz.tsinghua.edu.cn  }%\\
 %\smallskip
\author{ Feng Zhao  \quad zhaof17@mails.tsinghua.edu.cn}
%\institute{TBSI}
\date{9/18/2020}

\begin{document}
\setbeamertemplate{caption}{\raggedright\insertcaption\par}
\newcommand{\myalert}[1][]{\color{red} #1}
\begin{frame}
  \titlepage
\end{frame}


\AtBeginSection[]{%
  \begin{frame}<beamer> 
  
  \tableofcontents[ 
    currentsubsection, 
    hideothersubsections, 
    sectionstyle=show/hide, 
    subsectionstyle=show/shaded/hide]
  \end{frame}
  \addtocounter{framenumber}{-1}% If you don't want them to affect the slide number
}
 % Uncomment these lines for an automatically generated outline.
 %\begin{frame}{Outline}
 % \tableofcontents
 %\end{frame}
 
\section{Introduction}
 
\begin{frame}{Overview}

%\begin{itemize}
 %\item 
 
 \begin{itemize}
 % \item Review on Generalized Linear Models
 \item Environment choices
  \item Popular packages in Python
  \begin{itemize}
  	\item \texttt{numpy}
  	\item 
  	\texttt{scipy}
  	\item 
  	\texttt{matplotlib} 
  \end{itemize}
  \item GitHub classroom
  \end{itemize}  
\end{frame}

\begin{frame}{Scientific Programming Tools}
\begin{itemize}
	\item Operating systems, containers and clusters
	\item Programming language
	\begin{itemize}
		\item interpreted language: Python
		\item compiled language: C, C++
	\end{itemize}
	\item Package manager for Python
	\begin{itemize}
		\item pip: \url{https://pypi.org}
		\item conda: \url{https://anaconda.org}
		\end{itemize}
\end{itemize}
In this course, conda is recommended.
\end{frame}
\begin{frame}{Tips for using conda}
\begin{itemize}
	\item Download: \url{https://mirrors.tuna.tsinghua.edu.cn/anaconda/archive/}
	\item Setup Mirror: \url{https://mirrors.tuna.tsinghua.edu.cn/help/anaconda/}
	\item Install packages: \texttt{conda install scipy matplotlib}
	\item Check your install: \texttt{python -c "import numpy; print(numpy.\_\_version\_\_)"}
\end{itemize}
\end{frame}
\section{Popular packages in Python}
\begin{frame}[fragile]{Numpy}
Numpy: n-dimensional array manipulation
\begin{block}{Code snippet}
create a vector of length 3 and compute its $\ell_2$ norm
\begin{lstlisting}[language=Python]
import numpy as np
a = np.array([1, 2, 3])
print(np.linalg.norm(a))
\end{lstlisting}

compute the eigenvalues of a square matrix:
\begin{lstlisting}[language=Python,firstnumber=4]
A = np.array([[1, 2], [3, 4]])
print(np.linalg.eig(A)[0])
\end{lstlisting}

compute
the summation of each row for a matrix
\begin{lstlisting}[language=Python, firstnumber=6]
A = np.array([[1, 2], [3, 4], [5, 6]])
print(np.sum(A, axis=1))
\end{lstlisting}

matrix product
\begin{lstlisting}[language=Python,firstnumber=8]
print(A @ np.array([1, 1]))
\end{lstlisting}
\end{block}
\end{frame}
\begin{frame}[fragile]{Scipy}
Scipy:  algorithms of applied mathematics
\begin{block}{Code snippet}
the pdf of normal distribution
\begin{lstlisting}[language=Python,firstnumber=9]
import scipy.stats
x = np.linspace(-3, 3)
y = scipy.stats.norm.pdf(x)
print(x, y)
\end{lstlisting}
\end{block}
\end{frame} 
\begin{frame}[fragile]{Matplotlib}
Matplotlib -- plotting experiment results
\begin{block}{Code snippet}
	sample data from Gaussian and draw histogram
	\begin{lstlisting}[language=Python,firstnumber=13]
	import matplotlib.pyplot as plt
	c = np.random.normal(size=1000)
	plt.hist(c, density=True)
	plt.plot(x, y)
	plt.show()
	\end{lstlisting}
\end{block}
\end{frame}
\begin{frame}{Summary}
\begin{itemize}
	\item numpy
	\item scipy
	\item matplotlib
\end{itemize}
Further reference: \url{https://cs231n.github.io/python-numpy-tutorial/}
\end{frame}
\begin{frame}{GitHub Classroom}
Places to submit your programming assignments

\begin{block}{Steps}
\begin{enumerate}
	\item Register an account for GitHub
	\item Use Invitation URL to get the starting code
	\item Upload your modification to your own workspace
	\item Check the Autograding; Should be \textcolor{green}{\checkmark}; No \textcolor{red}{X} mark
\end{enumerate}
\end{block}
\end{frame}
\begin{frame}{Have a try}
\begin{block}{Linear regression}
Consider the linear observation model
\begin{equation*}
\bm{y} = X\bm{w} + \bm{c}
\end{equation*}
where the $X$ is a $10000 \times 10$ matrix, and $\bm{w},\bm{c}$ are column vectors with length 10 and 10000.
Use programming to find the $a$ that minimizes the loss $\frac12 \|X\bm{w}-y\|^2_2$. See details in the \textbf{linear\_regression.py}.
\end{block}
\begin{itemize}
\item Invitation URL: \url{https://classroom.github.com/a/ylEoHU6G}
\item Hint: use the formula: $\bm{w} = (X^TX)^{-1} X^T\bm{y}$.
\end{itemize}
\end{frame}
\end{document}
